\AgdaHide{
\begin{code}%
\>[0]\AgdaKeyword{module}\AgdaSpace{}%
\AgdaModule{CloTT.TypeFormers.Force}\AgdaSpace{}%
\AgdaKeyword{where}\<%
\\
%
\\[\AgdaEmptyExtraSkip]%
\>[0]\AgdaKeyword{open}\AgdaSpace{}%
\AgdaKeyword{import}\AgdaSpace{}%
\AgdaModule{Data.Product}\<%
\\
\>[0]\AgdaKeyword{open}\AgdaSpace{}%
\AgdaKeyword{import}\AgdaSpace{}%
\AgdaModule{Prelude}\<%
\\
\>[0]\AgdaKeyword{open}\AgdaSpace{}%
\AgdaKeyword{import}\AgdaSpace{}%
\AgdaModule{Presheaves.Presheaves}\<%
\\
\>[0]\AgdaKeyword{open}\AgdaSpace{}%
\AgdaKeyword{import}\AgdaSpace{}%
\AgdaModule{CloTT.Structure}\<%
\\
\>[0]\AgdaKeyword{open}\AgdaSpace{}%
\AgdaKeyword{import}\AgdaSpace{}%
\AgdaModule{CloTT.TypeFormers.Later}\<%
\\
\>[0]\AgdaKeyword{open}\AgdaSpace{}%
\AgdaKeyword{import}\AgdaSpace{}%
\AgdaModule{CloTT.TypeFormers.ClockQuantification}\<%
\\
%
\\[\AgdaEmptyExtraSkip]%
\>[0]\AgdaKeyword{open}\AgdaSpace{}%
\AgdaModule{PSh}\<%
\\
\>[0]\AgdaKeyword{open}\AgdaSpace{}%
\AgdaModule{■}\<%
\\
\>[0]\AgdaKeyword{open}\AgdaSpace{}%
\AgdaModule{►Obj}\<%
\end{code}
}

Finally, we show how to interpret \AIC{force}. To this aim, we
introduce an auxiliary function \AF{sem-force'}, which, given a type
\AB{A} and an inhabitant \AB{t} of \F{■}(\F{►} \Ar{A}), returns an
term in \F{■} \AB{A}.  For the field \AFi{■cone} of \AF{sem-force'}
\Ar{A} \AB{t}, we are required to construct an element of \AFi{Obj}
\AB{A} \AB{i} for each size \AB{i}.
Notice that \AFi{■cone} \AB{t}, when applied to a size
\Ar{i'}, gives a term \AB{t'} of type \F{►Obj} \AB{A} \AB{i'}. Furthermore, the
component \AFi{►cone} of \AB{t'}, when applied to a size \AB{j'} smaller
than \AB{i'}, returns a term of type \AFi{Obj} \AB{A} \AB{j'}.
Hence, in order to construct the required inhabitant of \AFi{Obj} \AB{A} \AB{i}, it suffices to find a size \AB{j} greater than \AB{i}.
An option for such a \AB{j} is \F{∞}. The field \AFi{■com} is defined in a similar way.
%is bigger than all sizes, we can define:

%%  To obtain such an element, we use that the type of \AB{t} involves both the box and the later modality.
%%Using the field \AFi{■cone} of \AB{t}, we get an term \AB{t'} of type \F{►Obj} \AB{A} \AB{i} for each size \AB{i}.
%%Note that we also have a function \AFi{►cone} \AB{t'} which gives for each size \AB{j} smaller than \AB{i} an \AFi{Obj} \AB{A} \AB{i}.
%%Hence, for an inhabitant of \AFi{Obj} \AB{A} \AB{i}, it suffices to find a size \AB{j} greater than \AB{i}.
%%Since \F{∞} is bigger than each size \AB{i}, we define
%%\remove{
%%Finally, we describe the interpretation of the term \AIC{force}.
%%We first define an auxilliary function \AF{sem-force'}.  Given a type
%%\AB{A} and an inhabitant \AB{t} of \F{■}(\F{►} \Ar{A}), this map gives an element of \F{■} \AB{A}.
%%For \AFi{■cone} (\AF{sem-force'} \AB{t}), we need to give an element of \AFi{Obj} \AB{A} for each size \AB{i}.
%%By definition, \AFi{■cone} \AB{t} \F{∞} has type \F{►Obj} \Ar{A} \F{∞}.
%%Since the size \F{∞} is bigger than \AB{i}, we can apply \Fi{►cone} (\Fi{■cone} \Ar{t} \F{∞}) to \IC{[} \Ar{i} \IC{]} to obtain an inhabitant of type \AFi{Obj} \AB{A}
%%\AB{i}.
%%We define the field \AFi{■com} of \F{sem-force'} similarly.
%%}
\begin{code}%
\>[0]\AgdaFunction{sem{-}force'}\AgdaSpace{}%
\AgdaSymbol{:}\AgdaSpace{}%
\AgdaSymbol{(}\AgdaBound{A}\AgdaSpace{}%
\AgdaSymbol{:}\AgdaSpace{}%
\AgdaFunction{SemTy}\AgdaSpace{}%
\AgdaInductiveConstructor{κ}\AgdaSymbol{)}\AgdaSpace{}%
\AgdaSymbol{→}\AgdaSpace{}%
\AgdaRecord{■}\AgdaSpace{}%
\AgdaSymbol{(}\AgdaFunction{►}\AgdaSpace{}%
\AgdaBound{A}\AgdaSymbol{)}\AgdaSpace{}%
\AgdaSymbol{→}\AgdaSpace{}%
\AgdaRecord{■}\AgdaSpace{}%
\AgdaBound{A}\<%
\\
\>[0]\AgdaField{■cone}\AgdaSpace{}%
\AgdaSymbol{(}\AgdaFunction{sem{-}force'}\AgdaSpace{}%
\AgdaBound{A}\AgdaSpace{}%
\AgdaBound{t}\AgdaSymbol{)}\AgdaSpace{}%
\AgdaBound{i}\AgdaSpace{}%
\AgdaSymbol{=}\AgdaSpace{}%
\AgdaField{►cone}\AgdaSpace{}%
\AgdaSymbol{(}\AgdaField{■cone}\AgdaSpace{}%
\AgdaBound{t}\AgdaSpace{}%
\AgdaPostulate{∞}\AgdaSymbol{)}\AgdaSpace{}%
\AgdaOperator{\AgdaInductiveConstructor{[}}\AgdaSpace{}%
\AgdaBound{i}\AgdaSpace{}%
\AgdaOperator{\AgdaInductiveConstructor{]}}\<%
\\
\>[0]\AgdaField{■com}\AgdaSpace{}%
\AgdaSymbol{(}\AgdaFunction{sem{-}force'}\AgdaSpace{}%
\AgdaBound{A}\AgdaSpace{}%
\AgdaBound{t}\AgdaSymbol{)}\AgdaSpace{}%
\AgdaBound{i}\AgdaSpace{}%
\AgdaBound{j}\AgdaSpace{}%
\AgdaSymbol{=}\AgdaSpace{}%
\AgdaField{►com}\AgdaSpace{}%
\AgdaSymbol{(}\AgdaField{■cone}\AgdaSpace{}%
\AgdaBound{t}\AgdaSpace{}%
\AgdaPostulate{∞}\AgdaSymbol{)}\AgdaSpace{}%
\AgdaOperator{\AgdaInductiveConstructor{[}}\AgdaSpace{}%
\AgdaBound{i}\AgdaSpace{}%
\AgdaOperator{\AgdaInductiveConstructor{]}}\AgdaSpace{}%
\AgdaOperator{\AgdaInductiveConstructor{[}}\AgdaSpace{}%
\AgdaBound{j}\AgdaSpace{}%
\AgdaOperator{\AgdaInductiveConstructor{]}}\<%
\end{code}
The semantic force operation follows immediately from \F{sem-force'}.
\begin{code}%
\>[0]\AgdaFunction{sem{-}force}\AgdaSpace{}%
\AgdaSymbol{:}\AgdaSpace{}%
\AgdaSymbol{(}\AgdaBound{Γ}\AgdaSpace{}%
\AgdaSymbol{:}\AgdaSpace{}%
\AgdaFunction{SemCtx}\AgdaSpace{}%
\AgdaInductiveConstructor{∅}\AgdaSymbol{)}\AgdaSpace{}%
\AgdaSymbol{(}\AgdaBound{A}\AgdaSpace{}%
\AgdaSymbol{:}\AgdaSpace{}%
\AgdaFunction{SemTy}\AgdaSpace{}%
\AgdaInductiveConstructor{κ}\AgdaSymbol{)}\AgdaSpace{}%
\AgdaSymbol{(}\AgdaBound{t}\AgdaSpace{}%
\AgdaSymbol{:}\AgdaSpace{}%
\AgdaFunction{SemTm}\AgdaSpace{}%
\AgdaBound{Γ}\AgdaSpace{}%
\AgdaSymbol{(}\AgdaRecord{■}\AgdaSpace{}%
\AgdaSymbol{(}\AgdaFunction{►}\AgdaSpace{}%
\AgdaBound{A}\AgdaSymbol{)))}\AgdaSpace{}%
\AgdaSymbol{→}\AgdaSpace{}%
\AgdaFunction{SemTm}\AgdaSpace{}%
\AgdaBound{Γ}\AgdaSpace{}%
\AgdaSymbol{(}\AgdaRecord{■}\AgdaSpace{}%
\AgdaBound{A}\AgdaSymbol{)}\<%
\\
\>[0]\AgdaFunction{sem{-}force}\AgdaSpace{}%
\AgdaBound{Γ}\AgdaSpace{}%
\AgdaBound{A}\AgdaSpace{}%
\AgdaBound{t}\AgdaSpace{}%
\AgdaBound{x}\AgdaSpace{}%
\AgdaSymbol{=}\AgdaSpace{}%
\AgdaFunction{sem{-}force'}\AgdaSpace{}%
\AgdaBound{A}\AgdaSpace{}%
\AgdaSymbol{(}\AgdaBound{t}\AgdaSpace{}%
\AgdaBound{x}\AgdaSymbol{)}\<%
\end{code}
