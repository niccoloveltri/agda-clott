\AgdaHide{
\begin{code}%
\>[0]\AgdaKeyword{module}\AgdaSpace{}%
\AgdaModule{CloTT.Structure.Types}\AgdaSpace{}%
\AgdaKeyword{where}\<%
\\
%
\\[\AgdaEmptyExtraSkip]%
\>[0]\AgdaKeyword{open}\AgdaSpace{}%
\AgdaKeyword{import}\AgdaSpace{}%
\AgdaModule{Prelude}\<%
\\
\>[0]\AgdaKeyword{open}\AgdaSpace{}%
\AgdaKeyword{import}\AgdaSpace{}%
\AgdaModule{Presheaves}\AgdaSpace{}%
\AgdaKeyword{public}\<%
\end{code}
}

Types are defined in a similar way.
Note that we are modelling a simple type theory and thus types do not depend on contexts.
For this reason, we can interpet types the same way as contexts.

\begin{code}%
\>[0]\AgdaFunction{Ty}\AgdaSpace{}%
\AgdaSymbol{:}\AgdaSpace{}%
\AgdaDatatype{ClockContext}\AgdaSpace{}%
\AgdaSymbol{→}\AgdaSpace{}%
\AgdaPrimitiveType{Set₁}\<%
\\
\>[0]\AgdaFunction{Ty}\AgdaSpace{}%
\AgdaInductiveConstructor{∅}\AgdaSpace{}%
\AgdaSymbol{=}\AgdaSpace{}%
\AgdaPrimitiveType{Set}\<%
\\
\>[0]\AgdaFunction{Ty}\AgdaSpace{}%
\AgdaInductiveConstructor{κ}\AgdaSpace{}%
\AgdaSymbol{=}\AgdaSpace{}%
\AgdaRecord{PSh}\<%
\end{code}


