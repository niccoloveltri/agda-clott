In dependent type theory, one usually works with terminating computations to ensure consistency.
Yet many interesting programs are not terminating and use possibly infinite data types.
To remain consistent, the computations must be productive \cite{Coquand93}.
This means: every finite part of an infinite object must be compute in a finite number of steps.

To allow convenient programming and reasoning with infinite objects, we need to extend our type theory with productivity checks.
In the previous years, various approaches were developed for such extensions.
One is based on sized types \cite{A-sized,AVW-normalization}, where types are assigned a size to indicate the number of unfoldings.
Another is based on guarded recursion \cite{atkey2013productive,BahrGM17}.

The relation between those approaches...

To compare these two approaches, we develop denotational semantics of guarded recursive type theory in a ttype theory with sized types.
More specifically, we define a presheaf model of a simple type theory with guarded recursion \cite{BMSS-synthetic}.
The interpretation of simple types are obtained via standard presheaf semantics.
However, guarded recursion and guarded recursive types are more difficult to obtain.
Since we use sizes rather than natural numbers, we cannot reuse the topos of trees \cite{BMSS-synthetic}.